\documentclass[9pt,twocolumn,times]{article}

\usepackage{color}
\usepackage{caption}
\usepackage{float}
\usepackage{courier}
\usepackage{xspace}
\usepackage{url}
\usepackage{listings}
\usepackage{graphicx}
\usepackage{mdframed}
\usepackage[letterpaper, margin=1in]{geometry}
\usepackage{enumitem}
\usepackage[compact]{titlesec}

\setitemize{noitemsep,topsep=0pt,parsep=0pt,partopsep=0pt,leftmargin=.5cm}
\setlist[itemize]{noitemsep,topsep=0pt,parsep=0pt,partopsep=0pt,leftmargin=.5cm}
\setlist[enumerate]{noitemsep,topsep=0pt,parsep=0pt,partopsep=0pt,leftmargin=.5cm}
\setlist[description]{noitemsep,topsep=0pt,parsep=0pt,partopsep=0pt,leftmargin=.5cm}

\lstset{
  language=C,
  basicstyle=\ttfamily\footnotesize,
  frame=lrbt,
  morekeywords={action,apply,bit,bool,%
const,control,default,else,%
enum,error,extern,exit,%
false,header,header_union,if,%
in,inout,int,match_kind,%
package,parser,out,return,%
select,state,struct,switch,%
table,transition,true,tuple%
typedef,varbit,verify,void,%
%
abstract,interface,class,virtual% used for IR
}
}

\newcommand{\code}[1]{\texttt{#1}}
\newcommand{\keyword}[1]{{\bf \texttt{#1}}}
\newcommand{\vonemodel}{\code{v1model}\xspace}
\newcommand{\mycomment}[1]{}

\title{Linux Network Programming with P4}
\author{William Tu\\
  VMware Inc.\\
  \texttt{tuc@vmware.com}
  \and
  Fabian Ruffy\\
  University of British Columbia\\
  \texttt{fruffy@cs.ubc.ca}
  \and
  Mihai Budiu\\
  VMware Research\\
  \texttt{mbudiu@vmware.com}
}
\date{}

\begin{document}
\maketitle

\begin{abstract}
  P4 is a domain-specific language for implementing network data-planes.
  The P4 abstraction allows programmers to write network protocols in a 
  generalized fashion, without needing to know the configuration specifics
  of the targeted data-plane.
  
  The extended Berkeley Packet Filter (eBPF) is a safe virtual machine for 
  executing sand-boxed programs in the Linux kernel. eBPF, and its extension 
  the eXpress Data Path (XDP), effectively serve as programmable data-planes of 
  the kernel.

  P4C-XDP is a project combining the performance of XDP with the generality and 
  usability of P4. In this document, we describe how P4 can be 
  translated into eBPF/XDP. We review the fundamental limitations of both 
  technologies, analyze the performance of several generated XDP programs, and 
  discuss problems we have faced while working on this new technology.
  
\end{abstract}


 \section{Introduction}\label{sec:introduction}

The introduction of Software Defined Networking (SDN)~\cite{rfc7426}
has decoupled the network control-plane from the data-plane.  The Open
Flow Protocol~\cite{mckeown-ccr08} is a typical incarnation of SDN.
Even though SDN makes the control-plane programmable, it still assumes
that the {data-plane} is fixed.  The inability to program data-planes
is a significant impediment to innovation: for example, the deployment
of the VxLAN protocol~\cite{rfc7348} took 4 years between the initial
proposal and its commercial availability in high-speed devices.

To address this state of affairs, \cite{bosshart-ccr14} introduced the
P4 language: Programming Protocol-independent Packet Processors, which
is designed to make the behavior of \emph{data-planes} expressible as
software.  P4 has gained rapid adoption.  The p4.org
consortium~\cite{p4org} was created to steward the language evolution;
p4.org currently includes more than 100 organizations in the areas of
networking, cloud systems, network chip design, and academic
institutions.  The P4 specification is open and
public~\cite{p416-spec17}.  Reference implementations for compilers,
simulation and debugging tools are available with a permissive license
at the GitHub P4 repository~\cite{p4lang}.  While initially P4 was
designed for programming network switches, its scope has been
broadened to cover a large variety of packet-processing systems (e.g.,
network cards, FPGAs, etc.).

\definecolor{grey}{rgb}{0.9, 0.9, 0.9}
\global\mdfdefinestyle{mdstyle}{%
	innerleftmargin=.3cm,rightmargin=.3cm,backgroundcolor=grey}
\begin{figure*}[h]
	\begin{mdframed}[style=mdstyle]
		\begin{description}
			\item[Headers] describe the format (the set of fields, their 
			ordering
			and sizes) of each header within a network packet.
			\item[User-defined metadata] are data structures associated with 
			each
			packet.
			\item[Intrinsic metadata] is information provided or consumed by the
			target, associated with each packet (e.g., the input port where a
			packet has been received, or the output port where a packet has to
			be forwarded).
			\item[Parsers] describe the permitted header sequences within 
			received
			packets, how to identify those header sequences, and the headers to
			extract from packets.  Parsers are expressed as state-machines.
			\item[Actions] are code fragments that describe how packet header
			fields and metadata are manipulated. Actions may include parameters
			supplied by the control-plane at run time (actions are closures
			created by the control-plane and executed by the data-plane).
			\item[Tables] associate user-defined keys with actions.  P4 tables
			generalize traditional switch tables; they can be used to implement
			routing tables, flow lookup tables, access-control lists, and other
			user-defined table types, including complex decisions depending on
			many fields.  At runtime tables behave as match-action
			units~\cite{bosshart-sigcomm13}, processing data in three steps:
			\begin{itemize}
				\item Construct lookup keys from packet fields or computed
				metadata,
				\item Perform lookup in a table populated by the control-plane,
				using the constructed key, and retrieving an action (including
				the associated parameters),
				\item Finally, execute the obtained action, which can modify the
				headers or metadata.
			\end{itemize}
			\item[Control] blocks are imperative programs describing the
			data-dependent packet processing including the data-dependent
			sequence of table invocations.
			\item[Deparsing] is the construction of outgoing packets from the
			computed headers.
			\item[Extern objects] are library constructs that can be manipulated
			by a P4 program through well-defined APIs, but whose internal
			behavior is hardwired (e.g., checksum units) and hence not
			programmable using P4.
			\item[Architecture definition:] a set of declarations that describes
			the programmable parts of a network processing device.
		\end{description}
	\end{mdframed}
	\caption{\sl Core abstractions of the P4 programming 
	language.\label{fig:abstractions}}
\end{figure*}

\begin{figure}[ht]
	\centerline{\includegraphics[width=.5\textwidth,clip,trim=1in 0.9in
		.8in 1.8in]{architecture.pdf}}
	\caption{\sl Generic abstract packet-processing engine programmable
		in P4.\label{fig:architecture}}
\end{figure}
\section{Background concepts}\label{sec:background}

This section briefly describes the P4 programming language and
eBPF. Some of the text is adapted from~\cite{budiu-osr17}
and~\cite{p4-ebpf-backend}.

\subsection{The P4 programming language}

\mycomment{(This section is adapted from~\cite{budiu-osr17}.)}
P4 is a relatively
simple, statically-typed programming language, with a syntax based on
C, designed to express transformations of network packets.

The core abstractions provided by the P4 language are listed in
Figure~\ref{fig:abstractions}.  P4 lacks many common features found in
other programming languages: for example, P4 has \textbf{no} support
for pointers, dynamic memory allocation, floating-point numbers, or
recursive functions; looping constructs are only allowed within
parsers.

P4 emphasizes static resource allocation; unlike systems such as the
Linux \texttt{tc} subsystem, in P4 all packet processing rules and all tables
must be declared when the P4 program is created.


\subsection{P4 Architectures}

P4 allows programs to execute on arbitrary \emph{targets}.  Targets
differ in their functionality, (e.g., a switch has to forward packets,
a network card has to receive or transmit packets, and a firewall has
to block or allow packets), and also in their custom capabilities
(e.g., ASICs may provide associative TCAM memories or custom checksum
computation hardware units, while an FPGA switch may allow users to
implement custom queueing disciplines).  P4 embraces this diversity of
targets and provides some language mechanisms to express it.


Figure~\ref{fig:architecture} is an abstract view of how a P4 program
interacts with the data-plane of a packet-processing engine.  The
data-plane is a fixed-function device that provides several
programmable ``holes''. The user writes a P4 program to ``fill'' each
hole.  The target manufacturer describes the interfaces between the
programmable blocks and the surrounding fixed-function blocks.  These
interfaces are target-specific.  Note that the fixed-function part can
be software, hardware, firmware, or a mixture.

A P4 architecture file is expected to contain declarations of types,
constants, and a description of the control and parser blocks that the
users need to implement.  Sections~\ref{sec:ebpf} and~\ref{sec:xdp}
contain examples P4 architecture description files.

\subsection{eBPF for network processing}

\mycomment{(This section is adapted from~\cite{p4-ebpf-backend}.) }
eBPF is an acronym that stands for Extended Berkeley Packet Filters. In essence,
eBPF is a low-level programming language (similar to machine code);
eBPF programs are traditionally executed by a virtual machine that
resides in the Linux kernel. eBPF programs can be inserted and removed
from a live kernel using dynamic code instrumentation. The main
feature of eBPF programs is their static safety: prior to execution,
all eBPF programs have to be validated as being safe, and unsafe
programs cannot be executed. A safe program provably cannot compromise
the machine it is running on:
\begin{itemize}
\item it can only access a restricted set of memory regions (verified
  either statically or through inline bounds checks using
  software-fault isolation techniques~\cite{wahbe:93}),
\item it can run only for a limited amount of time; during execution
  it cannot block, sleep or take any locks,
\item it cannot use any kernel resources with the exception of a
  limited set of kernel services which have been specifically
  whitelisted, including operations to manipulate tables (described
  below)
\end{itemize}

\begin{table*}[t]
	\footnotesize
	\begin{center}
		\begin{tabular}{|l|l|l|} \hline
			\textbf{Limitation} & \textbf{P4} & \textbf{eBPF} \\ \hline \hline
			Loops & Parsers & Tail call \\ \hline
			Nested headers & Bounded depth & Bounded depth \\ \hline
			Multicast/broadcast & External & Helpers \\ \hline
			Packet segmentation & No & No \\ \hline
			Packet reassembly &	No & No \\ \hline
			Timers/timeouts/aging & External & No \\ \hline
			Queues & No & No \\ \hline
			Scheduling & No & No \\ \hline
			Data structures & No & No \\ \hline
			Payload processing & No & No \\ \hline
			State & Registers/counters & Maps \\ \hline
			Iterating over packet payload & No & No \\ \hline
			Wildcard matches & Yes & No \\ \hline
			Table/map writes & Control-plane only & Data-plane and 
			control-plane \\ \hline
			Iteration over table/map values & Control-plane only & 
			Control-plane only \\ \hline
			Synchronization (data/data, data/control)  & No & No \\ \hline
			Resources & Statically allocated & Limited stack and buffer \\ 
			\hline
			Control-plane support & Complex, including remote & Simple \\ \hline
			Safety & Safe & Verifier limited to small programs \\ \hline
			Compiler & Target-dependent & LLVM back-end \\ \hline
		\end{tabular}
		\caption{Comparison of the limitations of P4 and
			eBPF.}\label{table:limitations}
	\end{center}
\end{table*}

\subsubsection{Kernel hooks}

eBPF programs are inserted into the kernel using hooks; their
execution is triggered when the flow of control reaches these hooks:


\begin{itemize}
\item function entry points can act as hooks; attaching an eBPF
  program to a function foo() will cause the eBPF program to execute
  every time some kernel thread executes foo().

\item eBPF programs can also be attached using the Linux Traffic
  Control (TC) subsystem, in the network packet processing
  datapath. Such programs can be used as TC classifiers and actions.

\item eBPF programs can also be attached to sockets or network
  interfaces. In this case, they can be used for processing packets
  that flow through the socket/interface.
\end{itemize}

\subsubsection{eBPF Maps}

The eBPF runtime exposes a bi-directional kernel- to user-space data
communication channel, called maps.  eBPF maps are essentially
key-value tables, where keys and values are arbitrary fixed-size
bitstrings.  The key width, value width and the maximum number of
entries that can be stored in a map are declared at map creation time.

In user-space, maps are are exposed as file descriptors. Both user- and
kernel-space programs can manipulate maps by inserting, deleting,
looking up, modifying, and enumerating entries.

In kernel-space, the keys and values are exposed as pointers to the raw
underlying data stored in the map, whereas in user-space the
pointers point to copies of the data.

\subsubsection{Concurrency}

The execution of an eBPF program is triggered by the corresponding
kernel hook; multiple instances of the same kernel hook can be running
simultaneously on different cores.

A map may be accessed simultaneously by multiple instances of the same
eBPF program running as separate kernel threads on different cores.
eBPF maps are native kernel objects, and access to the map contents is
protected using the kernel RCU mechanism. This makes access to table
entries safe under concurrent execution; for example, the memory
associated to a value cannot be accidentally freed while an eBPF
program holds a pointer to the respective value.  However, accessing
maps is prone to data races; since eBPF programs cannot use locks,
some of these races often cannot be avoided.

\subsection{XDP: eXpress Data Path}\label{sec:xdp-background}

An XDP program is a specific class of eBPF packet processing program, which 
attaches to the lowest levels of the networking
stack~\cite{xdpconext18}. XDP was initially designed to prevent 
denial-of-service attacks by quickly deciding whether a packet should be 
dropped, before too many kernel resources have been allocated. An XDP program 
can inspect a network packet right after the DMA engine copies the packet
from the network card and can also access eBPF maps.

The kernel expects an XDP program to return the decision taken about the 
processed packet. This may be one of the following values:

\begin{description}
\item[XDP\_DROP] the packet should be immediately dropped,
\item[XDP\_TX] bounce the received packet back on the same port it arrived on,
\item[XDP\_PASS] continue to process the packet using the normal kernel network 
stack,
\item[XDP\_REDIRECT] forward the packet to another port.
\end{description}

\subsection{Comparison of P4 and eBPF}

A very thorough evaluation of eBPF for writing networking programs can
be found in~\cite{minao-hspr18}.  P4 and eBPF share many features.
Table~\ref{table:compare} shows a comparison of the high-level
features of both languages.

\begin{table}[t]
  \footnotesize
  \begin{center}
  \begin{tabular}{|p{1.5cm}|p{2.7cm}|p{2.7cm}|} \hline
    \textbf{Feature} & \textbf{P4} & \textbf{eBPF} \\ \hline \hline
    Targets & ASIC, software, FPGA, NIC & Linux kernel \\ \hline
    Licensing & Apache & GPL \\ \hline
    Tools & Compilers, simulators & LLVM back-end, verifier \\ \hline
    Level & High & Low \\ \hline
    Safe  & Yes & Yes \\ \hline
    Safety & Type system & Verifier \\ \hline
    Resources & Statically allocated & Statically allocated \\ \hline
    Policies & Match-action tables & Key-value eBPF maps \\ \hline
    Extern helpers & Target-specific & Hook-specific \\ \hline
    Execution model & Event-driven & Event-driven \\ \hline
    Control-plane API & Synthesized by compiler & eBPF maps \\ \hline
    Concurrency & No shared R/W state & Maps are thread-safe \\ \hline
  \end{tabular}
  \caption{Feature comparison between P4 and eBPF.}\label{table:compare}
  \end{center}
\end{table}

P4 was designed as a language for programming switching devices,
working mostly at levels L2 and L3 of the networking stack.  While P4
is being used for programming network end-points, e.g., smart NICs,
some end-point functionality cannot be naturally expressed in P4
(e.g., TSO, encryption, deep packet inspection).


Many of these P4 limitations are shared with eBPF.  In general, while
P4 and eBPF are good for performing relatively simple packet
filtering/rewriting, neither language is good enough to implement a
full end-point networking stack.  Table~\ref{table:limitations}
compares the limitations of P4 and eBPF.

\section{Compiling P4 to eBPF}\label{sec:compilation}

In this section we describe two open-source compilers that translate
P4 programs in stylized C programs, that can in turn be compiled into
eBPF programs using the LLVM eBPF back-end.


\subsection{Packet filters with eBPF}\label{sec:ebpf}

The eBPF back-end is part of the P4 reference compiler
implementation~\cite{p4-ebpf-backend}.  This back-end targets a
relatively simple packet filter architecture.
Figure~\ref{fig:ebpf-model} shows the architectural model of an eBPF
packet filter expressed in P4.  This architecture comprises a parser
and a control block; the control block must produce a Boolean value
which indicates whether the packet is accepted or not.

\begin{figure}[h]
\begin{lstlisting}
#include <core.p4>

extern CounterArray {
    CounterArray(bit<32> max_index, bool sparse);
    void increment(in bit<32> index);
}

extern array_table {
    array_table(bit<32> size);
}

extern hash_table {
    hash_table(bit<32> size);
}

parser parse<H>(packet_in packet, out H headers);
control filter<H>(in H headers, out bool accept);

package ebpfFilter<H>(parse<H> prs,
                      filter<H> filt);
\end{lstlisting}
\caption{Packet filter P4 architectural model for an eBPF
  target.}\label{fig:ebpf-model}
\end{figure}

Figure~\ref{fig:count} shows a P4 program written for this
architecture. It counts the number of IPv4 packets that are
encountered.

\begin{figure}
\begin{lstlisting}
#include <core.p4>
#include <ebpf_model.p4>

typedef bit<48> EthernetAddress;
typedef bit<32> IPv4Address;

header Ethernet {
   EthernetAddress dstAddr;
   EthernetAddress srcAddr;
   bit<16> etherType;
}

// IPv4 header without options
header IPv4 {
   bit<4>       version;
   bit<4>       ihl;
   bit<8>       diffserv;
   bit<16>      totalLen;
   bit<16>      identification;
   bit<3>       flags;
   bit<13>      fragOffset;
   bit<8>       ttl;
   bit<8>       protocol;
   bit<16>      hdrChecksum;
   IPv4Address  srcAddr;
   IPv4Address  dstAddr;
}

struct Headers {
   Ethernet eth;
   IPv4     ipv4;
}

parser prs(packet_in p, out Headers headers) {
   state start {
      p.extract(headers.eth);
      transition select(headers.eth.etherType) {
         0x800 : ip;
         default : reject;
      }
   }

   state ip {
      p.extract(headers.ipv4);
      transition accept;
   }
}

control pipe(in Headers headers, out bool pass){
   CounterArray(32w10, true) ctr;

   apply {
      if (headers.ipv4.isValid()) {
         ctr.increment(headers.ipv4.dstAddr);
         pass = true;
      } else
         pass = false;
   }
}

// Instantiate main package
ebpfFilter(prs(), pipe()) main;
\end{lstlisting}
\caption{A P4 program that counts the number of IPv4 packets
  encountered.}\label{fig:count}
\end{figure}

Compilation to C is fairly straightforward; the generated C program is
always memory-safe, using bounds-checks for all packet accesses.  For
the entire P4 program a single C function is generated which returns a
Boolean value.  Table~\ref{table:translation} shows how each P4
construct is converted to a C construct.  Currently, programs with
parser loops are rejected, but a parser loop unrolling pass (under
development) will allow such programs to be compiled.

\begin{table}[h]
  \footnotesize
  \begin{tabular}{|l|p{4.8cm}|} \hline
    \textbf{P4 construct} & \textbf{C Translation} \\ \hline \hline
    \texttt{header} & \texttt{struct} with an additional \texttt{valid} bit \\ 
    \hline
    \texttt{struct} & \texttt{struct} \\ \hline
    parser state    & block statement \\ \hline
    state transition & \texttt{goto} statement \\ \hline
    \texttt{extract} call & load/shift/mask data from packet \\ \hline
    table & 2 eBPF maps --- one for actions, one for the default action \\ 
    \hline
    table key type & \texttt{struct} type \\ \hline
    \texttt{action} & tagged \texttt{union} with action parameters \\ \hline
    \texttt{action} parameters & \texttt{struct} \\ \hline
    \texttt{action} body & block statement \\ \hline
    table \texttt{apply} & lookup in eBFP map + \texttt{switch} statement with 
    all actions \\ \hline
    counters & eBPF map \\ \hline
  \end{tabular}
  \caption{Compiling P4 constructs to C.}\label{table:translation}
\end{table}

\subsection{Packet forwarding with XDP}\label{sec:xdp}


A second P4 to C compiler, P4C-XDP, is available as an open-source
project hosted at~\cite{p4-xdp-backend}. P4C-XDP is licensed under the GNU 
GPL and Apache License.
This compiler extends the eBPF compiler from Section~\ref{sec:ebpf}. It can 
target either a packet filter, or a packet switch.  The following listing shows 
the XDP architectural model targeted by this compiler. You can see that a
P4 XDP program can (1) return to the kernel one of the four outcomes
described in Section~\ref{sec:xdp-background}, and (2) it can also
modify the packet itself, by inserting, modifying or deleting headers.
\section{Testing eBPF programs}\label{sec:testing}

To test the P4 to C compilers we have adapted and extended the
existing P4 testing infrastructure.  The infrastructure can perform
both functional correctness testing at the user level, and complete
end-to-end testing running programs in the kernel.

\subsection{User-Space Testing}

User-space testing validates the correctness of the code generated by
compiler and can be performed even on systems that lack eBPF support
in the kernel.  The user space testing framework does not depend on
the LLVM~\cite{llvm} or any particular kernel version.  It also does
not require usage of \texttt{iproute2}~\cite{iproute} tooling such as
\texttt{tc} or \texttt{ip}.  It is also easier to debug failing tests
in user-space, by using tools such as GDB~\cite{gdb},
Valgrind~\cite{valgrind}, or Wireshark~\cite{wireshark}.


\subsection{The Simple Test Framework}
\begin{figure}[h]
  \centering
  \includegraphics[width=\linewidth]{stf}
  \caption{Annotated example of a Simple Testing Framework (STF)
    program for testing the P4 compiler.}
  \label{fig:stf}
\end{figure}

The P4 compiler includes a simple language (STF = Simple Testing
Framework) to describe input/output packets and to populate P4 tables.
Initially STF was used with software simulators to validate P4
programs, but we have adapted it for testing the eBPF and XDP
back-ends.  The STF framework is written in Python.
Figure~\ref{fig:stf} shows a small program written in the STF
language.  Table \ref{table:stf} describes the list of currently
supported STF operations in the eBPF testing backend.

The STF \texttt{packet} statement describes an input port and the
contents of a inbound packet.  The \texttt{expect} statement describes
an output port and the contents of an outbound packet.

Tables can be populated using the \texttt{add} statement, which
indicates a P4 table and an action to insert in the table, including
values for the action parameters.  Currently we assume that all
\texttt{add} statements are executed prior to all the packet
manipulation statements.

Our testing framework converts \texttt{packet} statements into PCAP
(Packet CAPture) files, one for each input port tested.  \texttt{add}
statements are converted into C programs that populate eBPF maps.

Although STF supports testing counters as well, our eBPF testing
framework does not yet support this feature.

\begin{figure*}
	\centering
	\includegraphics[width=\linewidth]{testing_workflow}
	\caption{Testing workflow for a P4-eBPF program. Environment and
		target are provided by the user.}
	\label{fig:p4_testflow}
\end{figure*}

\subsection{The Test Runtime}

\begin{table}[h]
	\footnotesize
	\begin{center}
		\begin{tabular}{|p{2.8cm}|p{4.3cm}|} \hline
			\textbf{Command} & \textbf{Description} \\ \hline \hline
			\textbf{packet} port data & Insert a frame of bytes
			\textit{data} into port \textit{port}.    \\ \hline
			\textbf{expect} port data & Expect a frame of bytes
			\textit{data} on port \textit{port}.  \\ \hline
			\textbf{add} tbl priority match action & Insert a
			match-action entry with key \textit{match} and action
			\textit{action} into table \textit{tbl}. \\ \hline
			\textbf{setdefault} tbl action & Set the default action for table
			\textit{tbl}. \\
			\hline
			\textbf{check\_counter} tbl key==n & Check if the value on
			the entry \textit{key} in counter table \textit{tbl} matches
			\textit{n}.  \\
			\hline
			\textbf{wait} & Pause for a second. \\ \hline
		\end{tabular}
		\caption{The STF command palette.}\label{table:stf}
	\end{center}
\end{table}

Executing a P4 eBPF test is done in five stages (Figure
\ref{fig:p4_testflow}):

\begin{enumerate}
\item\textbf{compile-p4:} Compile the P4 file to C program.
\item \textbf{parse-stf:} Convert the STF file to a C program and into
  input PCAP files.
\item \textbf{compile-data-plane:} Compile and load the C programs
  into an executable.
\item \textbf{run:} Wire up the executable to read from the input PCAP
  files; run the executable -- first populate tables then execute the
  program over the input packets.  Capture the produced output packets
  into output files.
\item \textbf{check-results:} Compare the output packets with the
  expected results.
\end{enumerate}

\noindent These five stages look slightly differently when testing in
user-space and in kernel-space.

In user-space, we use a hash-table library to emulate eBPF maps.

When testing in kernel-space we compile the eBPF/XDP programs to eBPF
object files using LLVM.  Before the eBPF/XDP program is loaded, the
framework creates a bridge running in a network namespace.
Namespace-based isolation allows us to run multiple tests in parallel.
Virtual interfaces are attached to the bridge.  The testing runtime
injects packets into the associated ports using raw sockets. The
output results are recorded by attaching Tcpdump~\cite{tcpdump} to
each output virtual interface.

\section{Experimental results}\label{sec:results}
\subsection{Testbed}
All of our performance results use a hardware testbed that consists of
two Intel Xeon E5 2440 v2 1.9GHz servers, each with 1 CPU socket and
8 physical cores with hyperthreading enabled.
Each target server has an Intel 10GbE X540-AT2 dual
port NIC, with the two ports of the Intel NIC on one server connected
to the two ports on the identical NIC on the other server.
We installed p4c-xdp on one server, the {\em target server}, and
attached the XDP program to the port that receives the packets.
The other server, the {\em source server}, generates packets
at the maximum 10~Gbps packet rate of 14.88~Mpps using the DPDK-based
TRex~\cite{trex} traffic generator.  The source server sends minimum
length 64-byte packets in a {\em single} UDP flow to one port of the
target server, and receives the forwarded packets on the same port.
At the target server, we use only one core to process all packets.
Every packet received goes through the pipeline specified in P4.

We use the sample P4 programs in the tests directory and the following
metrics to understand the performance impact of the P4-generated XDP
program:
\begin{itemize}
\item Packet Processing Rate (Mpps): Once the XDP program finishes
  processing the packet, it returns one of the actions mentioned in
  section~\ref{sec:background}.  When we want to count the number of
  packets that can be dropped per second, we modify each P4 program to
  always return XDP\_DROP.
\item CPU Utilization: Every packet processed by the XDP program is run
  under the per-core software IRQ daemon, named
  \texttt{ksoftirqd/\textit{core}}.  All packets are processed by only
  one core with one kthread, the ksoftirqd, and we measure the CPU
  utilization of the ksoftirqd on the core.
\item Number of BPF instructions verified: For each program, we list
  the complexity as the number of BPF instructions the eBPF
  verifier scans.
\end{itemize}

The target server is running Linux kernel 4.19-rc5 and for all our
tests, the BPF JIT (Just-In-Time) compiler is enabled and JIT hardening
is disabled. All programs are compiled with clang 3.8 with llvm 5.0.
For each test program, we use the following
command from iproute2 to load it into kernel:
\begin{lstlisting}[frame=none]
ip link set dev eth0 xdp obj xdp1.o verb
\end{lstlisting}

The Intel 10GbE X540 NIC is running the \texttt{ixgbe} driver with 16 RX queues
set-up. Since the source server is sending single UDP flow, packets
always arrive at a single queue ID.  As a result, we collect the number
of packets being dropped at this queue.

\subsection{Results}

To compute the baseline performance we wrote two small XDP programs by
hand: \texttt{SimpleDrop}, drops all packets by returning
\texttt{XDP\_DROP} immediately.  \texttt{SimpleTX} forwards the packet
to the receiving port returning \texttt{XDP\_TX}.  Each of these
programs consists of only two BPF instructions.

\begin{lstlisting}[frame=none]
    /* SimpleDrop */
    0: (b7) r0 = 1 // XDP_DROP
    1: (95) exit

    /* SimpleTX */
    0: (b7) r0 = 3 // XDP_TX
    1: (95) exit
\end{lstlisting}

After, we attached the following P4 programs to the receiving device:
\begin{itemize}
\item xdp1.p4: Parse Ethernet/IPv4 header, deparse it, and drop.
\item xdp3.p4: Parse Ethernet/IPv4 header, lookup a MAC address
in a map, deparse it, and drop.
\item xdp6.p4: Parse Ethernet/IPv4 header, lookup and get a new TTL value
from eBPF map, set to IPv4 header, deparse it, and drop.
\item xdp7.p4: Parse Ethernet/IPv4/UDP header, write a pre-defined source port
and source IP, recalculate checksum, deparse, and drop.
\item xdp11.p4: Parse Ethernet/IPv4 header, swap src/dst MAC address,
deparse it, and send back to the same port (XDP\_TX).
\item xdp15.p4: Parse Ethernet header, insert a customized 8-byte header,
deparse it, and send back to the same port (XDP\_TX).
\end{itemize}

\begin{table}
\centering
\small
\begin{tabular}{llll}
  \underline{P4 program} & \underline{CPU Util.} & \underline{Mpps} & \underline{Insns./Stack}\\
  SimpleDrop & 75\% & 14.4 & 2/0 \\
  SimpleTX & 100\% & 7.2 & 2/0 \\
  xdp1.p4 &  100\% &  8.1 & 277/256 \\
  xdp3.p4 &  100\% &  7.1 & 326/256 \\
  xdp6.p4 &  100\% &  2.5 & 335/272 \\
  xdp7.p4 &  100\% &  5.7 & 5821/336 \\
  xdp11.p4 &  100\% &  4.7  & 335/216 \\
  xdp15.p4 &  100\% &  5.5 & 96/56\\
\end{tabular}
\caption{\footnotesize Performance of XDP program generated by
  p4c-xdp compiler using single core.}
\label{tab:perf}
\end{table}

As shown in Table~\ref{tab:perf}, xdp1.p4 allows us to measure the
overhead introduced by parsing and deparsing: a drop from 14.4~Mpps to
8.1~Mpps.  xdp3.p4 reduces the rate by another million PPS due to the
eBPF map lookup (this operation always returns NULL, no value from the
map is accessed).  xdp6.p4 has significant overhead because it
accesses a map, finds a new TTL value, and writes to the IPv4 header.
Interestingly, although xdp7.p4 does extra parsing to the UDP header
and checksum recalculation, it has only a moderate overhead because of
the lack of map accesses.

Finally, xdp11.p4 and xdp15.p4 show the transmit (XDP\_TX)
performance.  Compared with xdp11, xdp15.p4 invokes the
\texttt{bpf\_adjust\_head} helper function to reset the pointer for extra
bytes.  It does not incur much overhead because there
is already a reserved space in front of every XDP packet frame.

\subsection{Performance Analysis}

To further understand the performance overhead of programs generated
by p4c-xdp, we started to brake down the CPU utilization. We used the Linux
perf tool on the process ID of the ksoftirqd that shows 100\%:
\begin{lstlisting}[frame=none]
perf record -p <pid of ksoftirqd> sleep 10
\end{lstlisting}


\noindent The following output shows the profile of xdp1.p4:
{\scriptsize
\begin{verbatim}
 83.19% [kernel.kallsyms] [k] ___bpf_prog_run
 8.14%  [ixgbe]           [k] ixgbe_clean_rx_irq
 4.82%  [kernel.kallsyms] [k] nmi
 1.48%  [kernel.kallsyms] [k] bpf_xdp_adjust_head
 1.07%  [kernel.kallsyms] [k] __rcu_read_unlock
 0.40%  [ixgbe]           [k] ixgbe_alloc_rx_buffers
\end{verbatim}
}

This confirms that most of the CPU cycles are spent on executing the
XDP program, \texttt{\_\_\_bpf\_prog\_run}, which caused us to investigate the
eBPF C code of xdp1.p4.

\begin{table}
\centering
\small
\begin{tabular}{llll}
  \underline{P4 program} & \underline{CPU Util.} & \underline{Mpps} & \underline{Insns./Stack}\\
  xdp1.p4 &  77\% &  14.8 & 26/0 \\
  xdp3.p4 &  100\% &  13 & 100/16 \\
  xdp6.p4 &  100\% &  12 & 98/40 \\
\end{tabular}
\caption{\footnotesize Performance of XDP program without deparser.}
\label{tab:perf2}
\end{table}

After commenting out the deparser C code, performance increases
significantly (see Table~\ref{tab:perf2}).  In the generated
code, the p4c-xdp compiler always writes back the entire packet
content, even when the P4 program does not modify any fields.  In
addition, the parser/deparser incur byte-order translation, e.g.,
htonl, ntohl.  This could be avoided by always using network
byte-order in P4 and XDP.  We plan to implement optimizations to
reduce this overhead.

\section{Lessons learned}\label{sec:conclusions}

In general, our development experience is mirrored by the lessons described
in~\cite{minao-hspr18} and ~\cite{bertin-netdev17}.

\paragraph{No multi-/broadcast support}
While XDP is able to redirect single frames it does not have the ability to 
clone and redirect multiple packets similar to \texttt{bpf\_clone\_redirect}. 
This makes development of more sophisticated P4 forwarding programs problematic.

\paragraph{The stack size is too small}
More complex generated XDP programs are rejected by the verifier despite their 
safeness. 
This is a particular challenge when attempting to implement network function 
chaining or advanced pipelined packet processing in a single XDP program. 

\paragraph{Generic XDP and TCP}
Our testing framework uses virtual Linux interfaces and generic 
XDP~\cite{genericxdp} to verify XDP programs. 
Unfortunately, we are unable to test TCP streams as the protocol is not 
supported by this driver~\cite{xdptcp}.
Any program loaded by generic XDP operates after the creation of
the \texttt{skb} and requires the original packet data. Since TCP clones every 
packet and passes the unmodifiable \texttt{skb} clone, generic XDP is
bypassed and never receives the datagram.

\paragraph{Using the libbpf user-space library}
Compilation of eBPF programs in user-space requires substantial 
effort. Many function calls and variables available in sample programs are not 
available as general C library. Any user trying to interact with the 
generated C code has to provide their own sources. Currently, P4C-XDP maintains 
copies of utilities from kernel code or various online sources. This is not a 
sustainable approach. We plan to integrate \texttt{libbpf} with our repository 
to control and manage the eBPF programs and maps.

\paragraph{Pinned eBPF maps in network namespaces}
When using eBPF programs in namespaces, maps that were pinned via \texttt{tc} 
do not persist across \texttt{ip netns exec} calls. As consequence, any 
program has to be run in a single continuous shell command. Example:
{\scriptsize
	\begin{verbatim}
bash -c "tc filter add ...; ls  /sys/fs/bpf/tc/globals"
	\end{verbatim}
}
Once \texttt{ip netns exec} has finished, the reference to the eBPF 
map and all its associated state disappear. The eBPF program, however, remains 
attached to the virtual interface, leading to inconsistent packet processing 
behavior.



\bibliography{top}
\bibliographystyle{acm}

\end{document}
